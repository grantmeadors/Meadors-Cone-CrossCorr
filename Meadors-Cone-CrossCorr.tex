\documentclass{article}
\usepackage{hyperref}
\usepackage{graphicx}
\usepackage{listings}

\lstset{breaklines=true}

\begin{document}
\title{CrossCorr parameter space double-cone \\ 
LIGO-T1600313}
\author{Grant David Meadors}
\date{\today}
%\affiliation{AEI Hannover/Golm}

\maketitle

CrossCorr's 3D parameter space (frequency $f$, projected semi-major axis $a \sin i$, time of ascension $t_\mathrm{asc}$) includes a double-cone structure around simulated signals. 
On this cone, test statististc $\rho$ values are significantly higher than background, but less than the maximum. 
This structure can be considered a long-range correlation. 
While the metric approximation should not be expected to hold on the cone, the cone equation is derivable from degeneracies in the signal model. 
We illustrate this equation with figures and compare results to the corresponding $X$-pattern found in TwoSpect.

\section{Introduction}

A. Neunzert is currently investigating $X$-structures [LIGO-G1600577] in the TwoSpect pattern space that have been recently documented \hyperref[https://dx.doi.org/10.1088/0264-9381/33/10/105017]{[\textit{Classical and Quantum Gravity} \textbf{31} (2016) 105017]}.
Figure~\ref{TwoSpectGraph} illustrates such an $X$.
This structure is prominent in both the test statistic, $R$, and extrapolated single-template $\log_{10} p$-value (latter shown).
Here we note how investigations into CrossCorr [\textit{Physical Review D} \textbf{91} (2015) 102005] show that an analogous, conical feature exists for the test statistic $\rho$ in the ($f$, $a \sin i$, $t_\mathrm{asc})$ paremeter space.
While this long-range correlated structure should not be expected to be predicted by the metric approximation about peak $\rho$, it is a predictable consequence of the signal model.
Similar structures can reasonably be expected from other binary continuous-wave searches.

Scripts and plots are hosted on Atlas:

\begin{center}
\url{<
https://www.atlas.aei.uni-hannover.de/~grant.meadors/LSC/ScoX1/2016/07/21-CrossCorr/
>}
\end{center}

\noindent They are also copied in the \texttt{plots} and \texttt{scripts} directories accompanying the source for this report.
Appendix~\ref{source_code_appendix} contains the source code, with automatic linebreaks provided by the \LaTeX~\texttt{listings} package.
\noindent CrossCorr settings 
are logged in 
\texttt{log\_crosscorr.txt}. 
For example,
$T_\mathrm{sft}$ = 840 s, \texttt{maxLag} = 840 s, to aid direct comparison with TwoSpect.
The scripts are
\begin{itemize}
    \item \texttt{wrapCrossCorr.py}\\
      (wrapper function, itself called as in \texttt{example-TwoSpect-like.txt})
    \item \texttt{libCallCC.py}\\
      (functions for the wrapper)
    \item \texttt{createHeatmap.py}\\
      (grapher)
\end{itemize}

\noindent Note that we plot $\Delta f_\mathrm{obs}$,

\begin{equation} 
\Delta f_\mathrm{obs} = \frac{2 \pi a_p}{P} f,
\end{equation}

\noindent with $a_p = (a \sin i)/c$ and orbital period $P = 2 \pi / \Omega$.
This convention facilitates easier comparison.

\begin{figure}
\begin{center}
\includegraphics[trim= 0 0 0 0, clip, width=0.80\paperwidth,keepaspectratio]{plots/DFvsFresultsProb-H1_pulsar-040.pdf}
\caption{
\url{<
https://ldas-jobs.ligo-wa.caltech.edu/~gmeadors/TwoSpect/2014/06/18/injections/graph-H1-40e-26-TSni/DFvsFresultsProb-H1_pulsar-040.pdf
>}
}
\label{TwoSpectGraph}
\end{center}
\end{figure}

\section{Illustrations}


\begin{table}
\begin{center}
\begin{tabular}{lr}
\textbf{Parameter} & \textbf{Value}\\
\hline
\texttt{Alpha} & 4.2756992385\\
\texttt{Delta} & -0.272973858335\\
\texttt{refTime} & 1245974416.0\\
\texttt{Freq} & 100.015\\
\texttt{h0} & 4e-25\\
\texttt{cosi} & 0.96724721146\\
\texttt{psi} & 5.71141493942\\
\texttt{phi0} & 4.11462142958\\
\texttt{orbitasini} & 1.44\\
\texttt{orbitEcc} & 0.0\\
\texttt{orbitTp} & 1245967532.76\\
\texttt{orbitPeriod} & 68023.82\\
\texttt{orbitArgp} & 0.0\\
\hline
\texttt{Tsft} & 840\\
\texttt{maxLag} & 840\\
\texttt{sqrtSx} & 4e-24\\
\hline
\textit{(duration)} & 1e6
\end{tabular}
\caption{Injection parameters}
\label{injection_table}
\end{center}
\end{table}



We take 2D slices through the 3D CrossCorr parameter space.
These dimensions are referenced by one-letter shorthand: F for $f$, A for $a_p$, T for $t_\mathrm{asc}$.
For ease of comparison, $\Delta f_\mathrm{obs} \propto a_p$ is actually graphed for A.
Each slice preserves F-A-T cyclic order, meaning that we graph the FA, AT, and TF planes along $xy$-axes.
Injection parameters are listed in Table~\ref{injection_table}; \texttt{lalapps\_Makefakedata\_v5} generated the injection.



\subsection{Slices through injection center}



\begin{figure}
\begin{center}
\includegraphics[trim= 0 0 0 0, clip, width=0.80\paperwidth,keepaspectratio]{plots/match-TS/FAresultsR-band-100-0.pdf}
\caption{
\url{<
https://www.atlas.aei.uni-hannover.de/~grant.meadors/LSC/ScoX1/2016/07/21-CrossCorr/match-TS/FAresultsR-band-100.0.pdf
>}
}
\label{FAcenterGraph}
\end{center}
\end{figure}
\begin{figure}
\begin{center}
\includegraphics[trim= 0 0 0 0, clip, width=0.80\paperwidth,keepaspectratio]{plots/match-TS/ATresultsR-band-100-0.pdf}
\caption{
\url{<
https://www.atlas.aei.uni-hannover.de/~grant.meadors/LSC/ScoX1/2016/07/21-CrossCorr/match-TS/ATresultsR-band-100.0.pdf
>}
}
\label{ATcenterGraph}
\end{center}
\end{figure}
\begin{figure}
\begin{center}
\includegraphics[trim= 0 0 0 0, clip, width=0.80\paperwidth,keepaspectratio]{plots/match-TS/TFresultsR-band-100-0.pdf}
\caption{
\url{<
https://www.atlas.aei.uni-hannover.de/~grant.meadors/LSC/ScoX1/2016/07/21-CrossCorr/match-TS/TFresultsR-band-100.0.pdf
>}
}
\label{TFcenterGraph}
\end{center}
\end{figure}

Figures~\ref{FAcenterGraph},~\ref{ATcenterGraph}, and~\ref{TFcenterGraph} respectively depict the $f$-$\Delta f_\mathrm{obs}$, $\Delta f_\mathrm{obs}$-$t_\mathrm{asc}$, and $t_\mathrm{asc}$-$f$ plane through the center of an injection. 
The FA plane shows an $X$-pattern, AT an ellitical peak, and TF another $X$ pattern with a different slope.
The maximum of each pattern is at the center of each graph (the injection site).

\subsection{Slices through offset planes}

Slices through 5-bin offsets are also shown (\textit{c.f.,} \texttt{note-on-offsets.txt}) in Figures~\ref{FAoffsetGraph},~\ref{AToffsetGraph},~\ref{TFoffsetGraph}, in the same order as before.
The 5-bin offset is equivalent to

\begin{itemize}
    \item $\delta f$ (frequency) = $4.7 \times 10^{-4}$ Hz,
    \item $\delta \Delta f_\mathrm{obs}$ (modulation depth) = $6.6 \times 10^{-4}$ Hz,
    \item $\delta t_\mathrm{asc}$ (time of ascension) = $537$ s.
\end{itemize}

\noindent The FA plane shows a hyperboloid with transverse axis parallel to the $f$-axis, the AT plane an open ring with maxima along an ellipse with transverse axis parallel to the $t_\mathrm{asc}$-axis, and the TF plane another hyperboloid with transverse axis parallel to the $f$-axis.

\begin{figure}
\begin{center}
\includegraphics[trim= 0 0 0 0, clip, width=0.80\paperwidth,keepaspectratio]{plots/match-offset-TS/FAresultsR-band-100-0.pdf}
\caption{
\url{<
https://www.atlas.aei.uni-hannover.de/~grant.meadors/LSC/ScoX1/2016/07/21-CrossCorr/match-offset-TS/FAresultsR-band-100.0.pdf
>}
}
\label{FAoffsetGraph}
\end{center}
\end{figure}

\begin{figure}
\begin{center}
\includegraphics[trim= 0 0 0 0, clip, width=0.80\paperwidth,keepaspectratio]{plots/match-offset-TS/ATresultsR-band-100-0.pdf}
\caption{
\url{<
https://www.atlas.aei.uni-hannover.de/~grant.meadors/LSC/ScoX1/2016/07/21-CrossCorr/match-offset-TS/ATresultsR-band-100.0.pdf
>}
}
\label{AToffsetGraph}
\end{center}
\end{figure}

\begin{figure}
\begin{center}
\includegraphics[trim= 0 0 0 0, clip, width=0.80\paperwidth,keepaspectratio]{plots/match-offset-TS/TFresultsR-band-100-0.pdf}
\caption{
\url{<
https://www.atlas.aei.uni-hannover.de/~grant.meadors/LSC/ScoX1/2016/07/21-CrossCorr/match-offset-TS/TFresultsR-band-100.0.pdf
>}
}
\label{TFoffsetGraph}
\end{center}
\end{figure}

\section{Interpretation}

The conic sections observed in Figures~\ref{FAcenterGraph} through~\ref{TFoffsetGraph} indicate that the three-dimensional surface of maximal $\rho$ is a double-cone, centered at the injection parameters.
Since the behavior of $\rho$ near the injection is well-described by the metric in the methods paper, let us focus on the extended cone.
When clarity is needed, true injection parameters are subscribed with zero (\textit{e.g.,} $f_0$).

\subsection{Heuristic approach}

Begin by considering the \textit{time-frequency} plane -- a spectogram, or plot of Short Fourier Transform bin powers over time -- of a solar-system barycentered signal.
Although power is concentrated in the extremal frequencies ($f_0 \pm \Delta f_\mathrm{obs}$) when averaged over time, as in the amplitude spectral density or an FFT over times of the \textit{time-frequency} plane, in the plane itself, the bins have comparitively uniform power.

Turning to the graphs: in the FA plane, the $X$-slope $d \Delta f_\mathrm{obs} / df = \pm1$.
This slope is identical to the $X$ in TwoSpect.
Its origin is described in depth in the TwoSpect directed methods paper.
When $f$ is offset by $\delta f$ and other parameters remain the same, the signal is lost.
However, if $\Delta f_\mathrm{obs}$ is also offset by $|\pm \delta \Delta f_\mathrm{obs}| = \delta f$, then the maximum (or minimum) frequency of the signal, $f_0 \pm \Delta f_\mathrm{obs}$, will be sensed by a \textit{template} (\textit{i.e.,} signal-model). 
Note also that when $\delta t_\mathrm{asc} \neq 0$, the true parameters miss the extrema; we deal with this more rigorously later.
In units of projected semi-major axis,

\begin{equation}
\frac{d (a \sin i)}{d f} = \frac{c P}{2 \pi f}.
\end{equation}

Observe next the TF plane.
By similar, phenomenological reasoning, we infer that the $X$-pattern is the result of contact between the rising (or falling) edges of the template with the true signal. 
Offset orbital phase $\delta t_\mathrm{asc}$ is compensated by shifting $\delta f$ so that the whole sinusoid moves parallel to the true signal.
The rising (or falling) edge of the \textit{time-frequency} sinusoid is caught when $\delta t_\mathrm{asc} = \pm \delta f / \max(df /dt)$.
When $\delta a_p \neq 0$, the true parameters miss the slope; we also deal with this more rigorously later.
Calculating $\pm \max \left(d f / d t\right)$ (identical results are obtained, up to a sign, by extremizing $t_\mathrm{asc}$),

\begin{eqnarray}
  \frac{df}{dt}
      &=& \frac{d}{dt} \left(f_0 + \frac{2 \pi f_0 a_p}{P} \sin \left[\Omega(t - t_\mathrm{asc})\right]\right),\\
      &=& \left(\frac{4 \pi^2 f_0 a_p}{P^2} \right) \cos[\Omega(t - t_\mathrm{asc})]). \\
  \max \left(\frac{df}{dt} \right)
      &=& \frac{4 \pi^2 f_0 a_p}{P^2},\\
      &\approx& 1.23 \times 10^{-6} \mathrm{~Hz~s}^{-1}, \nonumber\\
      &~& \mid {f_0 = 100.015\mathrm{~Hz}, a_p = 1.44\mathrm{~s}, P = 68023.82\mathrm{~s} },\\
      &\approx& 1\mathrm{~mHz} / (1000\mathrm{~s}).
\end{eqnarray}

\noindent Such a slope is observed in Figure~\ref{TFcenterGraph}.

Last, the ellipse in the AT plane emerges from moving the point of contact between the template and the signal.
For $\delta f = 0$, variations $\delta \Delta f_\mathrm{obs}$ and $\delta t_\mathrm{asc}$ will cause signal loss.
If $\delta f \neq 0$, then the signal is already lost.
We have already explained how $\delta \Delta f_\mathrm{obs} = \pm \delta f$ or $\delta t_\mathrm{asc} = \pm \delta f / \max (df/dt)$ restore contact, respectively at the extremal frequencies $f = f_0 \pm \Delta f_\mathrm{obs}$ or at the rising and falling edges, $f = f_0$.
We see in Figure~\ref{AToffsetGraph} that any point can be chosen on the ellipse with semi-major axes $(\delta \Delta f_\mathrm{obs}, \delta t_\mathrm{asc}) = (\delta f, \delta f/ \max (df /dt))$.

In sum, we describe the CrossCorr structure as a cone, give by the following parametric equations:

\begin{eqnarray}
s \in (-\infty, +\infty),~\theta \in [0, 2 pi);
\label{parametrics}
\end{eqnarray}

\begin{eqnarray}
\delta f                        &=& s, \label{fEq}\\
\delta \Delta f_\mathrm{obs} &=& s \times 1 \times \cos(\theta), \label{aEq}\\
\delta t_\mathrm{asc}                  &=& s \times \left(\max\left(\frac{df}{dt} \right)\right)^{-1} \sin(\theta) \label{tEq},
\end{eqnarray}

\noindent Equation~\ref{aEq} could equivalently be written

\begin{equation}
\delta a_p= s \times \frac{P}{2 \pi f_0} \times \cos(\theta).
\label{aEqFull}
\end{equation}

Likewise, Equation~\ref{tEq} expands to

\begin{equation}
\delta t_\mathrm{asc}    = s \times \frac{P^2}{(4 \pi^2 f_0 a_p)} \times \sin(\theta).
\label{tEqFull}
\end{equation}

\subsection{Periodicity in time of ascension}


At the largest scale, $t_\mathrm{asc}$ is periodic: $\rho(t_\mathrm{asc}) = \rho(t_\mathrm{asc} \mathrm{~mod~} P)$.
While we neglect this periodicity in our current studies, we expect that the full TF plane would show that the surface follows

\begin{equation}
\delta f = \max \left(\frac{df}{dt} \right) \frac{P}{2 \pi} \sin \left(\frac{P}{2 \pi} \delta t_\mathrm{asc} \right),
\end{equation}

\noindent of which the $\theta = \pi/2$ cases of Equations~\ref{fEq} and~\ref{tEq} are an approximation near $\delta t_\mathrm{asc} =0$.

\subsection{Alternate derivation from small errors}

Our problem can be approached more rigorously by considering small errors: $\epsilon_f$, $\epsilon_{a_p}$, $\epsilon_{t_\mathrm{asc}}$.
These correspond to the aforementioned offsets $\delta f$, $\delta a_p$, and $\delta t_\mathrm{asc}$.
Beginning from the signal model in the \textit{time-frequency} plane:


\begin{eqnarray}
f & = & \left(f_0 + \epsilon_f\right) \left( 1 + \frac{2 \pi (a_p + \epsilon_{a_p})}{P}\sin \left(\Omega (t - (t_\mathrm{asc}+\epsilon_{t_\mathrm{asc}}) \right) \right)\\
                   & \approx & f_0 + \epsilon_f \nonumber \\
                   &+& \frac{2\pi f_0}{P} \left[
                     (a_p+\epsilon_{a_p}) \sin(\Omega(t-t_\mathrm{asc}))
                   - a_p \Omega \epsilon_{t_\mathrm{asc}} \cos(\Omega(t-t_\mathrm{asc})) \right]  \nonumber \\
                   &+& \frac{2 \pi \epsilon_f}{P}  a_p  \sin(\Omega ( t-t_\mathrm{asc})).
\end{eqnarray}

The presence of the $\epsilon_f$ term independent in the first line, independent of $(t-t_\mathrm{asc})$, explains why $f$ is qualitatively different from $\Delta f_\mathrm{obs}$ and $t_\mathrm{asc}$, such that $f$ is the axis of the cone.
Simplifying by writing $\phi = \Omega (t- t_\mathrm{asc})$,

\begin{eqnarray}
f &\approx& f_0 + \epsilon_f \nonumber \\
                   &+&\frac{2 \pi f_0}{P}\left[
                     (a_p
                   + \epsilon_{a_p}) \sin \phi
                   - \frac{2 \pi  a_p}{P} \epsilon_{t_\mathrm{asc}} \cos \phi \right] \nonumber \\
                   &+& \mathcal{O}(0) \label{simplePhi}\\
  &\approx& \left(f_0 + \frac{2\pi f_0 a_p}{P} \sin \phi \right) \nonumber \\
  &+& \left(\epsilon_f + \frac{2 \pi f_0}{P} \epsilon_{a_p} \sin\phi - \frac{4 \pi^2 f_0 a_p}{P^2} \epsilon_{t_\mathrm{asc}} \cos\phi\right)
\label{niceApprox}
\end{eqnarray}

We rediscover earlier results in the ratios.
Let $E_f$, $E_a$, $E_t$ be the respective $\epsilon_f$, $\epsilon_{a_p}$, $\epsilon_{t_\mathrm{asc}}$ that individually yield a given $\max_\phi (f)$ when the other $\epsilon =0$:

\begin{eqnarray}
\frac{E_a }{ E_f }  &=& \frac{P }{2 \pi f_0} \label{errorA}\\
\frac{E_t }{ E_f }  &=& \frac{P^2}{4 \pi^2 f_0 a_p}.\label{errorT}\\
\frac{E_a}{E_t} &=& \frac{2 \pi f_0 a_p}{P}.\label{errorAT}
\end{eqnarray}

\noindent Not coincidentally, Equation~\ref{errorA} equals the ratio of Equation~\ref{aEqFull} to Equation~\ref{fEq} (when $\sin \theta = 1$).
Analogously, Equation~\ref{errorT} equals the ratio of Equation~\ref{tEqFull} to Equation~\ref{fEq} (when $\cos \theta = 1$).
Consider that if $|\cos \phi| = 1$, then $\Omega (t-t_\mathrm{asc}) = n \pi$ for integer $n$.
That is when $f \approx f_0$.
When $|\sin \phi| = 1$, then $f \approx f \pm \Delta f_\mathrm{obs}$.
This substantiates the earlier discussion of where each effect dominates.

Using the trigonometric identity that $a \sin x + b\sin (x+\alpha) = c\sin(x + \beta)$ with 

\begin{eqnarray}
c &=& \sqrt{a^2 + b^2 + 2 a b \cos \alpha}\\
\beta  &=& \arctan(b \sin \alpha / (a + b \cos \alpha)),
\end{eqnarray}

\noindent we can simplify the Equation~\ref{niceApprox} into a more revealing form:

\begin{eqnarray}
f &\approx& \left(f_0 + \frac{2\pi f_0 a_p}{P} \sin \phi \right) \nonumber\\
  &+& \epsilon_f  + \sqrt{\left(\frac{2 \pi f_0}{P} \epsilon_{a_p}\right)^2 + \left(\frac{4 \pi^2 f_0 a_p}{P^2} \epsilon_{t_\mathrm{asc}}\right)^2 } \sin \left(\phi + \beta \right). \label{altDefinition}\\
\beta &\equiv& \arctan\left(\frac{2\pi f_0 a_p}{P} \frac{\epsilon_{t_\mathrm{asc}}}{\epsilon_{a_p}} \right).
\label{angleOfContact}
\end{eqnarray}

\noindent These equations implicitly define the double-cone as the surface where the second line of Equation~\ref{altDefinition} sums to zero.

\subsection{Additional features in parameter space}

By including smaller terms, we could likely discover smaller, additional structures, such as the extended $W$ outside the central $X$ in TwoSpect.
The $W$ appears at $\delta f = \pm (\Delta f_\mathrm{obs} + \delta \Delta f_\mathrm{obs})$.
Equivalently, it is expected at

\begin{equation}
\delta f = \pm \frac{2\pi}{P} (a_p + \delta a_p) f_0.
\label{expectedW}
\end{equation}

\noindent Although there is a suggestive parallel in the interaction between line 1, term 2 and line 2, term 1 in Equation~\ref{altDefinition}, we must keep looking.
CrossCorr, unlike TwoSpect, is sensitive to orbital phase. 
Shifting $f$ by $\Delta f_\mathrm{obs}$, all else being equal, causes the template to be $\pi$ out of phase with the true signal.
Hence, the lines of Equation~\ref{expectedW} are only likely to be found if  $\delta a_p \approx P/2$.

\subsection{Point of contact on sinusoid}

Equation~\ref{angleOfContact} specifies exactly where the true signal and the CrossCorr signal model overlap.
At $\beta = n \pi$, it is at $f = f_0 \pm \Delta f_\mathrm{obs}$; at $\beta = n \pi + (\pi/2)$, it is at $f = f_0$.
Hence, up to a sign, $\beta = \theta$ from Equation~\ref{parametrics}.

\subsection{Validity of approximations}
To check the validity of approximating by $\mathcal{O}(0)$ in Equation~\ref{simplePhi}, restore the term,

\begin{eqnarray}
f  &\approx& \left(f_0 + \frac{2\pi f_0 a_p}{P} \sin \phi \right) \nonumber \\
  &+& \left(\epsilon_f + \frac{2 \pi}{P} (f_0 \epsilon_{a_p} + \epsilon_f a_p) \sin\phi - \frac{4 \pi^2 f_0 a_p}{P^2} \epsilon_{t_\mathrm{asc}} \cos\phi\right)
\label{betterApprox}\\
 &\approx& \left(f_0 + \frac{2\pi f_0 a_p}{P} \sin \phi \right) \nonumber\\
  &+& \epsilon_f  + \sqrt{\left(\frac{2 \pi}{P} (f_0 \epsilon_{a_p} + \epsilon_f a_p )\right)^2 + \left(\frac{4 \pi^2 f_0 a_p}{P^2} \epsilon_{t_\mathrm{asc}}\right)^2 } \sin \left(\phi + \beta \right). \label{bestDefinition}\\
\beta &\equiv& \arctan\left(\frac{2\pi f_0 a_p}{P} \frac{\epsilon_{t_\mathrm{asc}}}{f_0 \epsilon_{a_p} + \epsilon_f a_p} \right).
\end{eqnarray}

\noindent For comparison, one can rearrange,


\begin{equation}
f_0 \epsilon_{a_p} + \epsilon_f a_p = \Delta f_\mathrm{obs}\left(\frac{\epsilon_{a_p}}{a_p} + \frac{\epsilon_f}{f_0} \right),
\end{equation}

\noindent
For Sco X-1 search configurations, $\epsilon_{a_p}/a_p$ tends to be much larger than $\epsilon_f /f_0$, justifying our earlier approximation.
More usefully, solve for zero sum at $\phi=0$, $\epsilon_{t_\mathrm{asc}} =0$:


\begin{eqnarray}
\epsilon_f + \frac{2\pi}{P} (f_0 \epsilon_{a_p} + \epsilon_f a_p) = 0,\\
\epsilon_f = -\frac{1}{1 - (2\pi a_p/P)} \frac{2\pi}{P} f_0 \epsilon_{a_p}.
\end{eqnarray}

\noindent Thus the precise error from our approximation in Equation~\ref{simplePhi} is a factor of $2 \pi a_p /P$ in the slope of the line in the FA plane (about $2\times 10^{-5}$ for Sco X-1).

\subsection{Eccentricity}

Pending detailed investigations, it seems a safe assumption that CrossCorr behaves similarly to TwoSpect in response to eccentricity.
Consult LIGO-T1500594.
Concisely, low eccentricity should merely shift the observed position of the signal without significantly weakening it.

\subsection{Background subtraction}

Each CrossCorr figure in the FA and TF planes shows a band of starkly-reduced $\rho$ at frequencies above or below the true $f_0$ (and beyond the conic structure).
This reduction in $\rho$ is presumably due to background subtraction over frequency bins, although this should be further investigated.
Moreover, the AT plane Figure~\ref{AToffsetGraph} shows a zone of reduced $\rho$ in the center.
As the entire graph is at an offset $f$, this central zone is probably reduced for the same reason.

\section{Conclusion}

Double-cone structures exist in the CrossCorr parameter space, and the basic features can be inferred mathematically.
This feature is a consequence of the signal model.

In the future, this feature might be exploitable.
Lines parallel to the $f$-axis, at sparsely-spaced $\Delta f_\mathrm{obs}$ and $t_\mathrm{asc}$, but densely spaced in $f$, could be probed.
This is convenient, since resampling needs to be done per each $\Delta f_\mathrm{obs}$ and $t_\mathrm{abs}$ but yields a dense array of $f$.
The $\rho$ values calculated along these lines could be assembled into a meta-statistic, much as $R$ for TwoSpect can be assembled into a $X$ statistic (as with A. Neunzert's work).
Another approach would be to use K. Wette's approach [LIGO-P1600162] and simulate a range of artificial metric mismatches with Monte Carlo, fit the true mismatches, and employ a much coarser, empirical grid.
Presumably, the meta-statistic and empirical grid should converge to the same result, though the former may be provably correct and the latter easier to implement, especially for high-dimensional spaces.
Either the meta-statistic or empirical grid might have acceptable detection efficiency: while not as good as $\rho$ itself, it may be much less computationally-costly, especially given the 3D nature of the CrossCorr search.
Since lag-time $\texttt{maxLag}$ is tunable, the computational savings could be reinvested in longer lag.
The precise benefits depend on the Receiver Operating Characteristic curve, as yet unknown, as a function of sparseness in $\Delta f_\mathrm{obs}$ and $t_\mathrm{asc}$.
An all-sky CrossCorr search is not inconceivable.

Meanwhile, for the full CrossCorr search, the emergence of the double-cone for a loud candidate or outlier could be considered evidence of a real signal, along the lines of \textit{signal morphology} tests in other groups.
Finally, knowing the extent of the double-cone and whether the sum of $\rho$ values converges could be helpful in determining the long-range correlations in the search and calculating a true $p$-value, adjusted by trials factor, for a candidate.


\newpage

\appendix
\section{Source code}
\label{source_code_appendix}

\subsection{example-TwoSpect-like.txt}
\lstinputlisting{scripts/example-TwoSpect-like.txt}
\subsection{wrapCrossCorr.py}
\lstinputlisting[language=Python]{scripts/wrapCrossCorr.py}
\subsection{libCallCC.py}
\lstinputlisting[language=Python]{scripts/libCallCC.py}
\subsection{createHeatmap.py}
\lstinputlisting[language=Python]{scripts/createHeatmap.py}


\end{document}
